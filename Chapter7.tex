% -*- Mode: latex; mode: linkd; mode: auto-fill; mode: flyspell;-*-
\chapter{Conclusion}
\label{chap-seven}

% Talk about 
% What are the implications of the 
This project began with very ambitious goals, but in the end most of
the original objectives were achieved.  This chapter discusses the
utility of the project, the challenges that were faced, and a way
to move forward on this work.

\section{Discussion}

Readers may wonder about the utility of this undertaking. We have some
confidence that this project will add value to the cognitive modeling
community. It provides a central location where students and
researchers can learn from one another; it improves support and
resources available for ambitious cognitive modeling projects that in
some cases might not be possible otherwise, due to constraints on
hardware or software infrastructure; finally it has the potential to
foster a stronger culture of collaboration among community members.

\subsection{Challenges and future work}

%Talk about the limitations of the system

Coglaborate has some limitations.
%
There exists a graphical user interface to ACT-R, which we have
mentioned only briefly.  Coglaborate, in its current state, does not
provide all of the same functionality of this interface.  For example,
Coglaborate does not enable the user to step through productions or
view the contents of buffers as a model is being executed.  The
existing graphical user interface may thus make it easier to build and
debug models.  However, Coglaborate offers compensating advantages,
and it could easily be extended to incorporate existing development
aids.
%  - Lack of a stepper
%  - Buffer, production viewer

% Persistence
Like BioBike, Coglaborate provides persistence that is only apparent;
objects persist over the lifetime of the active Lisp environment.  The
practical limitation here is the lack of standardization between
database systems that support persistence and Lisp environments in
common use.  Recently Allegro has introduced an object persistence
library called Allegro Cache as part of its platform; adding
persistence in the future should be a simple matter of programming.

%  - As of now no clean way to provide support for module that require
%  visual support except through means of collaboration.

%  - System needs a bit of polish, a few minor bugs
%  - A better way to provide isolation apart from meta processes
%    - A means by which we could spawn new lisp processes that could
%    sandbox the whole environment.
The current version of the system uses an abstraction of ACT-R called
meta processes to provide isolation between models, although as of now
it serves its purpose, true isolation could be achieved by permitting
the application server let every user spawn a new Lisp process if
required. This facility would make the system more robust and might
help load balancing is needed in the future.

Other limitations remain to be discovered.  We believe that the system
can best be tested for effectiveness by opening it up the community,
obtaining the feedback and improving it. This way we can test the
effectiveness of collaboration in cognitive modeling in the real
world.

\subsection{Implications of this work}

On creating a frame-based abstraction for ACT-R models it quickly
became clear that this representation could be used to explore a
number of other possibilities. This section discusses two such
possibilities.

% Talk about what areas of research this opens up and the implications
As observed by Langley et al.~\cite{citeulike:4182324}, an important
issue facing cognitive modeling is support for software reuse. This
project promotes reuse of models in the sense that the representation
allows for models to be represented, analyzed, and distributed in a
much more transparent fashion than in their current representation as
Lisp code.  Today, it is impossible to determine the similarity
between two ACT-R models except through code inspection and ad hoc
judgments.  The frame-based representation introduced in this research
makes more sophisticated analysis possible: comparison of the use of
buffers across productions, for example.  Such analyses remain for
future work.

Another interesting research direction would be to investigate software
reuse as provided by object-oriented programming environments.  That
is, we can develop features such that models can inherit behavior from
other more general models.  This way we should be able identify
general patterns that emerge from human cognition.

Yet another area of interest for this system is the investigation of
user interfaces that allow cognitive scientists to create models
without actually having to learn Lisp. This could lead to a gentler
learning curve for people working with ACT-R.
% Software reuse as in an object oriented way. This way we would
% be able to get more general models and hence theories that would
% help us identify patterns

% Collaboration - 

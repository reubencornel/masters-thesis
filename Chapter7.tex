% -*- Mode: latex; mode: linkd; mode: auto-fill; mode: flyspell;-*-
\chapter{Conclusion}
\label{chap-seven}

% Talk about 
% What are the implications of the 
When we decided to embark on this project we set pretty high goals for
our selves. We ended up achieving most of the original objectives of
the project. This chapter discusses the utility of the project, the
challenges we faced and tries to envision a way forward for this project.

\section{Discussion}

The readers of this report may inquire about the utility of this
undertaking. We can say with confidence that this project would add
tremendous value to the community. It would provide a central location
where student and researchers can learn from one another; it would
provide the support and resources need to carry out ambitious
cognitive modeling projects that in some cases might not be possible
due to constraints on hardware or software infrastructure and finally
it would foster a strong culture of collaboration the community.

\subsection{Challenges and future work}

%Talk about the limitations of the system

Unlike the graphical interface to ACT-R, Coglaborate does not provide
the user with the ability to step through productions and view the
contents of the buffers as the model is being executed. This as a
result may make it difficult to deal with bugs in models. 
%  - Lack of a stepper
%  - Buffer, production viewer

% Persistence
Like BioBike, Coglaborate provides apparent persistence. This is
primarily because existing database systems do not permit us to store
lisp objects to the disk. But recently Allegro has introduced an
object persistence library called Allegro Cache as part of its
platform. Therefore adding persistence in the future would be trivial.

%  - As of now no clean way to provide support for module that require
%  visual support except through means of collaboration.

%  - System needs a bit of polish, a few minor bugs
%  - A better way to provide isolation apart from meta processes
%    - A means by which we could spawn new lisp processes that could
%    sandbox the whole environment.
The current version of the system uses an abstraction of ACT-R called
meta processes to provide isolation between models, although as of now
it serves its purpose, true isolation could be achieved by permitting
the application server let every user spawn a new lisp process if
required. This facility would make the system more robust and might
help in case we might need to perform load balancing operations.

The system can be tested for effectiveness by opening out to the
community, obtaining the feedback and improving it. This way we can
test the effectiveness of collaboration in cognitive modeling in the
real world.


\subsection{Implications of this work}

When creating a representation for ACT-R models I realized that we
could use this representation to open up a number of other
possibilities. This section discusses two such possibilities.

% Talk about what areas of research this opens up and the implications
As observed by Langley et. al. in \cite{citeulike:4182324} one issue
facing cognitive modeling is the issue of providing software
reuse. This project promotes reuse of models but in perhaps the most
naive manner possible. A more interesting research problem would be to
investigate software reuse as provided by object oriented programming
systems. That is we can develop features such that models can inherit
behavior from other more general models, this way we would be able
identify general patterns that emerge from human cognition.

Another area of interest that we envision for this system would be
investigation into creating a user interface that would allow
cognitive scientists to create models with out actually having the
learn lisp. This way we can provide a gentler learning curve for
people working with ACT-R.
% Software reuse as in an object oriented way. This way we would
% be able to get more general models and hence theories that would
% help us identify patterns

% Collaboration - 

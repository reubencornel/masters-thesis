% -*- mode: latex; mode: linkd; mode: auto-fill -*-
% Index
% (@> "Introduction")
% (@> "Advantages provided by cogworks")

\chapter{Collaborative modelling}
\label{chap-four}

% (@file :file-name "thesis.tex")

% (@* "Introduction")
The objective of Cogworks is to provide a collaborative modelling
environment for the cognitive science community. It does so by
mounting ACT-R on the Biobike infrastructure. The system currently
supports collaboration asynchronously. This means that a certain
researcher can develop a model and share it with his colleague, who
can work on that model separately and share it with other individuals.

This chapter discusses the advantages provided by a shared environment
such as Cogworks. It also tries to provide a framework for the future
development of Cogworks, by providing the means that we used to come
up with the requirements for this system.

\section{Collaboration}

%explain why is collaboration required. 
Collaboration is the key to building large structures in almost all
human endeavors be it in the fields of either the arts or
sciences. Collaboration permits breaking down large unwieldy tasks
into more manageable tasks. It also permits sharing of knowledge and
resources. Computer networks now provide us with a means by which we
can help teams across geographical regions collaborate effectively.

% I'm talking about collaboration, I could read up more about
% collaboration and put that in.

% Briefly describe the main types of systems that help in
% collaborative work

\section{Advantages}
% (@* "Advantages provided by cogworks")

\subsection{Remote Access}

\subsection{Access to different knowledge bases}

\subsection{Data Sharing}

\section {Organization of Cogworks}

\subsection{Tasks}

\subsection{Senarios}

%Do I need this section?
\subsection {Requirements}
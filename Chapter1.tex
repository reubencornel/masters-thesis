\chapter{Introduction}
\label{chap-one}

%Things to talk about here

\section {Cognitive Architectures}
\label{introCogArch}
% Introduce cognitive modelling
The study of the human mind has been one of the largest and most
laborious enterprises undertaken by man. This study has encompassed many
centuries and has attracted interest from some of the most brilliant
philosophers and scientists human kind has had to offer such as
Aristotle, Plato, Descartes and more recently Allen Newell, Marvin
Minksy. 

%% Introduce computer science into the picture
%%  Introduce the various ways by which we have tried to under stand
The advent of computers allowed researchers to use the superior
processing power to simulate the working of the
mind. Some researchers attempted to use a purely connectionist model to
%TODO: Cite Rosenblatt
represent the working of the human mind. Where as
others, such as Newell and Anderson attempted to use the manipulations
of symbols partially relying on an underlying connectionist model to
represent the same.  

%% Introduce cognitive modeling 
% Introduce the need for architectures
Despite large advances made towards understanding the principles of
the human mind in the field of psychology during the 70s, there was little to show
except for micro theories, such as Fitts law and the Power law of
%TODO: cite unified theories of cognition here..
practice, etc. Newell argues in his book (cite unified theories
of cognition) that the aim of psychology is to provide a framework
that would help explain and predict the behavior of the human mind, and although micro
theories are useful in explaining certain phenomena they do not provide a
framework that explains and predicts the behavior of the human
mind. He advocates the need for a comprehensive theory that would fill
in this void and attempts to do so by providing a framework called SOAR. 
%TODO: cite unified theories of cognition here..

Another cognitive architecture was introduced by John Anderson from
the Carnegie Mellon University. He was working on a framework that
tried to provide a working model of the 
human memory called ACT*. As this work progressed, new modules to
represent motor and visual faculties were added to this
framework. This later came to be known as ACT-PM then ACT-R. 

\section {Collaboration}
% Talk about collaboration
%% In Collaboration we can also talk about ...
%% 
Computer networks allowed researchers
to investigate the ability to use computers to help people collaborate
with one another. The original idea was to use computers to replace
physical apparatus such as white boards in meeting rooms (cite Colab). 
Researchers discovered paradigms that could be used for to support
collaborations such as, synchronous and asynchronous collaboration. As
more work was done in this field the tools such as PREP(cite Issues in
design computer support for co-authoring and commenting.) a tool that
supported interaction between authors and commentors,
Cognoter(Theory and practice of a colab-orative tool) a tool that
provided a framework to help the decision making process, etc.

Despite these strides in the area of computer supported collaborative
work, researchers in the cognitive modeling community 
shared models through means of conferences, summer schools and
personal websites. The objective of my research is to provide a
collaborative environment where cognitive scientists can share
models, I describe this further in the following section and chapters.

\section{Contributions}

%Talking about contributions
% What are the contributions
% 1) Describe the fact that we are providing a collaborative
% environment
I would like to start this section by acknowledging that the overall
direction for the research was provided by 
Dr. St. Amant. The aim of this project is to provide a collaborative
environment for researchers who develop computational representations
of their work. This is achieved by allowing researchers to access
ACT-R on top of a web based framework where they can create 
and share model. The benefits of this project are:
%    The advantages of this are 
%    -- a) We are providing an environment with a complete setup 
%       b) Collaboration - sharing of knowledge
%       c) Hardware resources

\begin{itemize}
\item We provide a software environment completely setup and ready for
  use. As a result researcher can get to work with out being concerned
  about issues regarding software dependencies.
\item We attempt to foster collaboration in the cognitive modeling
  community. Researchers can build models and these can be accessed by
  other individuals and groups, and can be modified and shared back.
\item By providing a centralized system hardware resources can be
  shared across a large number of users. This would make it inexpenive
  to run and be of use to research groups that cannot invest in
  hardware.
\item Since users store their models on a centralized
  system it acts as a repository for models that can be used to
  learn about cognitive modeling.
\end{itemize}


% 2) The means used to achieve this. 
% TODO cite the minsky paper.
%   Using frames...
%      The advantages of this.
I used frames(cite the minsky paper) to achieve collaboration
. A frame is a data structure consisting of slots, where each
slot can either be another frame or an instance of data. Apart from
achieving a means to provide collaboration, this data structure lets
us represent models in a structured manner. This opens up a number of
avenues for research, the first one that does come to mind is the
ability to mine models to find patterns in different models and
hence build up a more complete picture of human cognition.

% Talk About the fact that there is no collaboration between researchers
% Talk about the difficulties with this 

% Introduce what this research intends achieve and perhaps the long
% term goals of the project.

\section{Thesis Organization}
In Chapter 2 I discuss cognitive architectures and the features of
various cognitive modeling 
architectures. Chapter 3 attempts to describe the field of computer supported
collaborative work. Chapter 4 aims to provide an overview of the
biobike system. Chapter 5 describes the implementation of the
system. In chapter 6 I describe how I solved the problem of a synonym
crossword using this system. We conclude this with a discussion of the
limitations of this system and the future work for this system.
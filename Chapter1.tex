\chapter{Introduction}
\label{chap-one}

%Things to talk about here

\section {Cognitive Architectures}
\label{introCogArch}
% Introduce cognitive modelling
The study of the human mind has been one of the largest and most
laborious enterprises undertaken by man. This study has encompassed
many centuries and has attracted interest from some of the most
brilliant philosophers and scientists human kind has had to offer,
from Aristotle, Plato, and Descartes to modern researchers in
cognitive science, including Allen Newell, John Anderson, and Marvin
Minksy.

%% Introduce computer science into the picture
%%  Introduce the various ways by which we have tried to under stand
The development of computers was an important step for theories of
mind, in that it allowed researchers to begin to simulate the
structure of thought.  Approaches vary.  Some researchers, such as
McClelland and Rumelhart~\cite{rumelhart1986researchv1,rumelhart1986researchv2}, have worked with
abstractions of low-level biological components of the
brain---artificial neurons---to produce connectionist models of the
mind.
%TODO: Cite Rosenblatt
Others, such as Newell and Anderson, have worked at a higher level, focusing
on the manipulation of symbols as a foundation for thinking.  Beyond
this broad division there have been many other approaches over the
past half-century.

%% Introduce cognitive modeling 
% Introduce the need for architectures
Despite large advances made towards understanding the workings of
the human mind in the field of psychology during the 70s, general
principles were mainly to be found in micro theories, such as Fitts'
law and the Power law of Practice.
%TODO: cite unified theories of cognition here..
In his influential book, {\em Unified Theories of
  Cognition}~\cite{Newell:1990aa}, Newell argues that the aim of
psychology is to provide a framework that would help explain and
predict the behavior of the human mind, and although micro theories
are useful in explaining certain phenomena they do not provide an
adequate framework for explaining and predicting human thinking in
more general terms.  He advocates the need for a comprehensive theory
that would fill in this void.
%TODO: cite unified theories of cognition here..

One such theory has been under development by Anderson since the
1970s.  It takes the form of a cognitive architecture, a computational
simulation of problem solving constrained by empirical findings
concerning human cognition.  Anderson began with a framework that
tried to provide a working model of the human memory, called ACT*.  As
this work progressed, new modules to represent motor and visual
faculties were added to this framework. This came to be known as
ACT-PM and eventuall ACT-R.  ACT-R provides the conceptual focus for
this thesis.

\section {Collaboration}
% Talk about collaboration
%% In Collaboration we can also talk about ...
%% 
A thriving research community has grown up around the core group of
ACT-R researchers at Carnegie Mellon University.  There are annual
workshops, a ``summer school'' to introduce new researchers to the
framework, an active mailing list, and any number of small
interdisciplinary groups of collaborators distributed throughout the
world.  The result has been a continuous stream of refinements to
ACT-R, both the theory and the software, as well as models,
experiments, development tools, and the like.

This provides an opportunity, one identified in a different area of
computer science, computer-supported collaborative work.
J. C. R. Licklider described his vision of what he called online
communities in 1968~\cite{licklider1968computer}:

\begin{quotation}
 [T]here are at present perhaps only as few as half a dozen
  interactive multiaccess computer communities. These communities are
  socio-technical pioneers, in several ways out ahead of the rest of
  the computer world: What makes them so? First, some of their members
  are computer scientists and engineers who understand the concept of
  man-computer interaction and the technology of interactive
  multiaccess systems. Second, others of their members are creative
  people in other fields and disciplines who recognize the usefulness
  and who sense the impact of interactive multiaccess computing upon
  their work. Third, the communities have large multiaccess computers
  and have learned to use them. And, fourth, their efforts are
  regenerative\ldots
 
 What will on-line interactive communities be like? In most
  fields they will consist of geographically separated members,
  sometimes grouped in small clusters and sometimes working
  individually. They will be communities not of common location, but
  of common interest. In each field, the overall community of interest
  will be large enough to support a comprehensive system of
  field-oriented programs and data.
\end{quotation}

A wide range of tools to support online collaboration has been
developed over the intervening years.
%
% Computer networks allowed researchers to investigate the ability to
% use computers to help people collaborate with one another. The
% original intention was to use computers to replace physical apparatus
% such as white boards in meeting rooms (cite Colab).  Researchers
% discovered paradigms that could be used for to support collaborations
% such as, synchronous and asynchronous collaboration. As more work was
% done in this field the tools such as PREP(cite Issues in design
% computer support for co-authoring and commenting.) a tool that
% supported interaction between authors and commentors, Cognoter(Theory
% and practice of a colab-orative tool) a tool that provided a framework
% to help the decision making process, etc.
%
The objective of my research is to provide a collaborative environment
where cognitive scientists can share models, to explore the potential
benefits of this approach to cognitive modeling.  I describe this
further in the following section and chapters.

\section{Contributions}

%Talking about contributions
% What are the contributions
% 1) Describe the fact that we are providing a collaborative
% environment
The aim of this project is to provide a collaborative environment for
researchers who develop computational representations of their
research on cognition, specifically in the ACT-R architecture.  This
is achieved by allowing researchers to access ACT-R on top of a
web-based framework where they can create and share models.  The
benefits of this project are as follows:
%    The advantages of this are 
%    -- a) We are providing an environment with a complete setup 
%       b) Collaboration - sharing of knowledge
%       c) Hardware resources

\begin{itemize}
\item We provide a software environment completely set up and ready
  for use. As a result, researchers can get to work without being
  concerned about software dependency issues, a traditional barrier in
  collaboration as well as in ACT-R modeling.
\item We attempt to foster collaboration in the cognitive modeling
  community. Researchers can build models and these can be accessed by
  other individuals and groups, and can be modified and shared in
  return.
\item By providing a centralized system, hardware resources can be
  shared across a large number of users. This should in principle make
  it inexpenive to run and be of use to research groups that cannot
  invest in the resources necessary for some cognitive modeling efforts.
\item Since users store their models on a centralized system, it acts
  as a repository for models that can be used to learn about cognitive
  modeling.
\end{itemize}


% 2) The means used to achieve this. 
% TODO cite the minsky paper.
%   Using frames...
%      The advantages of this.
Frames~\cite{Minsky1974a} are used as the central representation to
support collaboration, at the sofware level.  A frame is a data
structure consisting of slots, where each slot can either be another
frame or an instance of data. Apart from a means to support
collaboration, this data structure lets us represent models in a
structured manner, something that has not been possible before with
ACT-R.  This opens up a number of further avenues for research, such
as the ability to mine models to find patterns in different models and
hence build up a more complete picture of human cognition.

% Talk About the fact that there is no collaboration between researchers
% Talk about the difficulties with this 

% Introduce what this research intends achieve and perhaps the long
% term goals of the project.

\section{Thesis Organization}
In Chapter 2 I discuss cognitive architectures and the features of
various cognitive modeling architectures. Chapter 3 gives a brief
overview of the field of computer-supported collaborative work,
concentrating on research relevant to this thesis.  Chapter 4 provides
an overview of the biobike system, a collaborative platform I have
adapted to cognitive modeling.  Chapter 5 describes the implementation
of the system I have built, called Coglaborate.  To demonstrate the
use of the system, as a proof of concept, I describe how I solved the
problem of a synonym crossword in chapter 6.  I conclude this thesis
with a discussion of the limitations of Coglaborate and future work
for the system.

\chapter{Introduction}
\label{chap-one}
% Let's start with a few paragraph basics, here is how to make \textbf{bold} , \textit{italics}, and \underline{underline}.  Let's say you need to cite something in your references, simply do this \cite{t06}.  Some compilers need you to compile something twice in order for citations and references to show up in the pdf.  Here is a quotation:
% \begin{quotation}
% Alice, Bob and Carol are boring.  Who would even want to know their secret?
% \end{quotation}

% Let's say we need to make a list, try this on for size
% \begin{enumerate}
% \item NCSU is great\\
% \item I like NCSU\\
% \item I really hope I can find a job when I graduate!
% \end{enumerate} 

% \section{Math enviroments}
% \subsection{Equations}

% There are many different ways to write equations, for example we could put $a^2 + b^2 = c^2$ directly into a sentence.  Or we could use the equation enviroment and do 

% \begin{equation}\label{eq.one}
% a^2+b^2=c^2.
% \end{equation} And from here we can later reference it by simply doing this (\ref{eq.one}).  If you don't need to reference an equation you may simply so this $$a^2 + b^2 = c^2.$$

% For Greek letters you must go to the math enviroments, for example $\alpha$, $\beta$, and $\gamma$.  Let's look at equations that cover multiple lines, none of these equations may be true or mean anything, but so that the reader can get some ideas.  In addition I will use some other useful notations like subscripts, superscripts, fractions, etc.  One important item of note is that one uses the ``ampersand" symbol to line up equations (also look at how I used quotations).

% \begin{eqnarray}
% \gamma_1 & = & \alpha^{\beta} + \psi_0 \frac{\psi_1}{\psi_2+\psi_3} \label{eq.two} \\
% & = & \beta_1 + \beta_2 + \ldots + \beta_k \nonumber\\
% & \rightarrow & E(\gamma_2) 
% \end{eqnarray}

% Alternatively, one can specify a slightly different enviroment if none of the equations need to be numbered.  Remember that if you are planning on referring to them later on, you must use a ``label" statement.

% \begin{eqnarray*}
% \gamma_1 & = & n^{-1/2} \displaystyle \sum_{i=1}^n \left[h(X_i,\beta_0)-E\{h(X_i,\beta_0)\}\right]\\
% & \rightarrow & \hat q \pm \frac{\partial \gamma_2}{\partial \beta}. 
% \end{eqnarray*}  Lastly there may be times in which you want to use a non-italicized word your formula, such as an indicator function that may look like this $\mbox{I}\{\mu_i(1,\beta)>\mu_i(0,\beta)\}$ , if so just use the ``mbox" statement.


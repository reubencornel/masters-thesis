\chapter{The nature of cogition}
\label{The_nature_of_cognition}

The quest to understand the working of the human mind has spanned
many centuries starting with Plato when he asked, as in the words of
Noam Chomsky citing Bertrand Russell, ``How comes is that human beings
whose contacts of the world are brief, personal and limited, are
nevertheless able to know as much as they do.'' \cite{Bogdan:1993aa}

Cognitive science brings together the varied disciplines of
psychology, neuroscience, computer science, linguistics and philosophy in an
attempt to answer the above question, using information processes as a
means to emulate the algorithms and processes of the human
mind. Psychology, especially cognitive psychology contributes theories
on cognitive capacities, information processing capabilities and
perhaps most importantly it tries to provide hypothesis that theorize
about the overall picture of the human mind. Neuroscience, the study
of the nervous system,  provides a frame of reference against which
theories developed in cognitive science can be validated since it
deals with the brain at the lowest level. Secondly, it provides  knowledge
for developing an alternative architecture of the mind. Computer
Science contributes to the enterprise knowledge representation which is
used to develop theories to represent the way knowledge is stored;
artificial intelligence which is used to analyse and create methods
for problem solving; the theory of computation which is used as a
means to develop representations for cognitive
architectures.

The objective of this chapter is three-fold; firstly it aims
to provide a very brief introduction to the human cognitive
architecture from both the cognitivist and
emergent\cite{DBLP:journals/tec/VernonMS07} perspective. Secondly
it discusses cognitive architecture in general and finally goes on
to compare some currently widely used cognitive architectures.

\section{The nature of cognition}
\label{nature_Of_Cognition}
Any attempt to deal with the architecture of cognition has to answer
the following questions.

\begin{itemize}
\item How is procedural and declarative knowledge acquired, and
represented?
\item How do various processes act on this knowledge and how do they
achieve the effect they intend to achieve?
\item How can these processes and structures be manifested in the real
world?
\end{itemize}



 When solving problems the human mind has the ability to
retrieve and apply previously stored knowledge to the problem; for
example consider solving a calculus based integration problem, we are
able to retrieve standard representations of the forms of equations
and apply it to the problem to simplify it and solve it. Hence this
question is one of the questions that is central to understanding
cognition.

This question is significant because its answer explains the
techniques of deduction and inference we use to solve problems on an
everyday basis; this could be as simple and routine as diagnosing a
light bulb is not working and replace it, or perhaps the techniques we
use when solving a crossword puzzle. 

This is summarized best by Bogdan \cite{Bogdan:1993aa}

\begin{quote}
It takes a real system, made of physical bits and pieces, to 
instantiate cognitive structures and processes and run the program of 
cognition.
\end{quote}

These questions provide us with a very general framework of the
results to be provided by cognitive science. Newell in his book
\emph{The Unified Theories of Cognition} \cite{Newell:1990aa} and in
\cite{Newell1980135} describes the study of the working of the mind as
a problem of satisfying the ``Conjunction of constraints on the nature
of mind like systems.'' He describes the characteristics of what is to
be expected of any theory that claims to propose a model of human
cognition. Newell mentions that this list is not comprehensive, but in
the view of Anderson \& Lebiere it can used to provide a broad
framework against which all theories that claim to explain the human
mind can be tested.
 

These criteria have been listed below, have been referenced from
\cite{CambridgeJournals:207162} and \cite{Newell:1990aa}. The purpose
of listing these criteria below is to explain as to what the study of
the mind would require.

\begin{itemize}
\item Behave flexibly as a function of the environment: At first
glance this statement describes that the nature of human cognition
is <CHOOSE A RIGHT WORD>. Newell did make it clear that he was referring
to the view that a cognitive system can be viewed as an instance of a
universal computer, specifically a turing machine, despite its occasional
failings and lack of infinite memory. He further explains that this
view does not indicate the inablity to perform special operations, for
example, vision. He explains that like computers with special
processing units the cognitive system can be made up of special
purpose systems that specialize in a certain task. As an example
consider the example of chemist they are able to perform congnitive
tasks that are relates to their field and they are also able to drive
their car. This 


\item Operate in real time: A system that models cognition should be
able to explain the reason as to how we are able to perform cognitive
tasks at the speed humans do. This criteria is important because if a
system is not able to explain it could lead us to wrong assumptions
about how humans think.


\item Exhibit rational adapative behaviour: It must be able to explain
this because humans perform computations and those computations, as in the words of
Newell\cite{Newell:1990aa}, are for ``the service of goals and
rationally related to obtaining things that let the organism survive
and propagate.''

\item Display dynamic behaviour: Humans operate in an
environment that is ever changing. They draw in this
information from their environment and act on it appropriately. For
example, if you are driving your car and at that moment a deer decides
to sprint in front of your car, you would hit the brakes. 

\item Integrate diverse knowledge: HELP REQUIRED HERE. Did not
understand what was being said in the papers to clearly explain what
is happening

\item Exhibits a sense of consciousness: Newell could not
point out to the direct relation between consciousness and human
cognition but he did mention it as one of the criteria in his tests of
human cognition. An interpretation of this as taken from
\cite{CambridgeJournals:207162} is that Newell was asking us to pick
out criteria for this test and the authors of that paper point towards
using sections from a volume titled ``Scientific approaches to
Consciousness".

\item Learning from the environment: This point should be self
evident, we gain new knowledge from the world around us. But then
the type of learning itself should be based on whether it can learn based on
semantic memory, skill, priming and conditioning.

\item Arise through evolution: HELP NEEDED HERE AGAIN.

\item Use of Natural language: Any theory that claims to decipher human
cognition must be able to explain as to how we are able to comprehend
what we listen, understand what we speak because this is a function
that is core to the way we communicate with each other.

\item Be realizable with in the brain: This point is critical because
it serves as proof that a given theory is congruous with actual
computations in the brain.
\end{itemize}


\subsection{Classical approach}
\subsubsection{Newells criteria}
\subsection{Connectionist approach}
\section{Cognitive Architectures}
\section{ACT-R}
\section{SOAR}
\section{EPIC}

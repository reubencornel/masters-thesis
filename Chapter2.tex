\chapter{The nature of cogition}
\label{The_nature_of_cognition}

The quest to understand the working of the human mind has spanned
many centuries starting with Plato when he asked, as in the words of
Noam Chomsky citing Bertrand Russell, ``How comes is that human beings
whose contacts of the world are brief, personal and limited, are
nevertheless able to know as much as they do.'' \cite{Bogdan:1993aa}

Cognitive science brings together the varied disciplines of
psychology, neuroscience, computer science, linguistics and philosophy in an
attempt to answer the above question, using information processes as a
means to emulate the algorithms and processes of the human
mind. Psychology, especially cognitive psychology contributes theories
on cognitive capacities, information processing capabilities and
perhaps most importantly it tries to provide hypothesis that theorize
about the overall picture of the human mind. Neuroscience, the study
of the nervous system,  provides a frame of reference against which
theories developed in cognitive science can be validated since it
deals with the brain at the lowest level. Secondly, it provides  knowledge
for developing an alternative architecture of the mind. Computer
Science contributes to the enterprise knowledge representation which is
used to develop theories to represent the way knowledge is stored;
artificial intelligence which is used to analyse and create methods
for problem solving; the theory of computation which is used as a
means to develop representations for cognitive
architectures.

The objective of this chapter is three-fold; firstly it aims
to provide a very brief introduction to the human cognitive
architecture from both the cognitivist and
emergent\cite{DBLP:journals/tec/VernonMS07} perspective. Secondly
it discusses cognitive architecture in general and finally goes on
to compare some currently widely used cognitive architectures.

\section{The nature of cognition}
\label{nature_Of_Cognition}
Any attempt to deal with the architecture of cognition has to answer
the following questions.

\begin{itemize}
\item How is procedural and declarative knowledge acquired, and
represented?
\item How do various processes act on this knowledge and how do they
achieve the effect they intend to achieve?
% TODO WORK ON THIS
\item How can these processes and structures be manifested in the real
world?
\end{itemize}

%# WARNING:CH2: ---
%# WARNING:CH2: Have ignored the faculties of perception, would that be a
%# WARNING:CH2: problem? The point is that I am abstracting away faculties
%# WARNING:CH2: because I believe they are problems that need to be
%# WARNING:CH2: sepately with out connection to the main project
%# WARNING:CH2: ---

%# TODO:CH2: Describe that each question is a set of solutions and not
%# TODO:CH2:  a single solution, due to the complex nature of the mind.

 When solving problems the human mind has the ability to
retrieve and apply previously stored knowledge to the problem; for
example consider solving a calculus based integration problem, we are
able to retrieve standard representations of the forms of equations
and apply it to the problem to simplify it and solve it. Hence this
question is one of the questions that is central to understanding
cognition.

This question is significant because its answer explains the
techniques of deduction and inference we use to solve problems on an
everyday basis; this could be as simple and routine as diagnosing a
light bulb is not working and replace it, or perhaps the techniques we
use when solving a crossword puzzle. 

%# TODO: CH2: Work on this section.
This is summarized best by Bogdan \cite{Bogdan:1993aa}

\begin{quote}
It takes a real system, made of physical bits and pieces, to 
instantiate cognitive structures and processes and run the program of 
cognition.
\end{quote}

These questions provide us with a very general framework of the results to
be provided by cognitive science. Newell in his book \emph{The
Unified Theories of Cognition} \cite{Newell:1990aa} and in \cite{Newell1980135} describes the
study of the working of the mind as a problem of satisfying the
``Conjunction of constraints on the nature of mind like systems.'' He
describes the 
characteristics of what is to be expected of any system that proposes
to implement a model of human cognition. The purpose of listing
these criteria is to elaborate on how the questions above map
into real cognitive systems. These criteria have
been listed below, have been referenced from
\cite{CambridgeJournals:207162} and \cite{Newell:1990aa}.

\begin{itemize}
\item Behave flexibly as a function of the environment: At first
glance this statement describes that the nature of human cognition
%#TODO: CH2: Choose good word
is <CHOOSE A RIGHT WORD>. Newell did make it clear that he was referring
to the view that a cognitive system can be viewed as an instance of a
universal computer, specifically a turing machine, despite its occasional
failings and lack of infinite memory. He further explains that this
view does not indicate the inablity to perform special operations, for
example, vision. He explains that like computers with special
processing units the cognitive system can be made up of special
purpose systems that specialize in a certain task. The solution to
this question would like within the set of solutions to the first
question. 

\item Operate in real time: A system that models cognition should be
able to explain the reason as to how we are able to perform cognitive
tasks at the speed humans do. This criteria is important because if a
system is not able to explain this then it leaves a void in the 

%Exhibit rational adapative behaviour
%Use knowledge about the environment
%Behave predictably and gracefully in the face of the
%Integrate diverseUse natural
%Exhibits self awareness and a sense of
%Learn from its
%Arise through
%Be realizable with in the brain.
\end{itemize}


\subsection{Classical approach}
\subsubsection{Newells criteria}
\subsection{Connectionist approach}
\section{Cognitive Architectures}
\section{ACT-R}
\section{SOAR}
\section{EPIC}

\chapter{Cognitive Architectures}
\label{chap-two}
The quest to understand the working of the human mind has spanned
many centuries starting with Plato when he asked, as in the words of
Noam Chomsky citing Bertrand Russell, ``How comes is that human beings
whose contacts of the world are brief, personal and limited, are
nevertheless able to know as much as they do.'' \cite{Bogdan:1993aa}

Cognitive science brings together the varied disciplines of
psychology, neuroscience, computer science, linguistics and philosophy in an
attempt to answer the above question, using information processes as a
means to emulate the algorithms and processes of the human
mind. Psychology, especially cognitive psychology contributes theories
on cognitive capacities, information processing capabilities and
perhaps most importantly it tries to provide hypothesis that theorize
about the overall picture of the human mind. Neuroscience, the study
of the nervous system,  provides a frame of reference against which
theories developed in cognitive science can be validated since it
deals with the brain at the lowest level. Secondly, it provides  knowledge
for developing an alternative architecture of the mind. Computer
Science contributes to the enterprise knowledge representation which is
used to develop theories to represent the way knowledge is stored;
artificial intelligence which is used to analyse and create methods
for problem solving; the theory of computation which is used as a
means to develop representations for cognitive
architectures.

The objective of this chapter is three-fold; firstly it aims
to provide a very brief introduction to the human cognitive
architecture from both the cognitivist and
emergent\cite{DBLP:journals/tec/VernonMS07} perspective. Secondly
it discusses cognitive architecture in general and finally goes on
to compare some currently widely used cognitive architectures.

\section{The nature of cognition}
\label{mindArch}

Any attempt to deal with the architecture of cognition has to answer
the following questions.

\begin{itemize}
\item How is procedural and declarative knowledge acquired, and
represented? When solving problems the human mind has the ability to
% WARNING:CH2: ---
% WARNING:CH2: Have ignored the faculties of perception, would that be a
% WARNING:CH2: problem? The point is that I am abstracting away faculties
% WARNING:CH2: because I believe they are problems that need to be
% WARNING:CH2: sepately with out connection to the main project
% WARNING:CH2: ---
retrieve and apply previously stored knowledge to the problem; for
example consider solving a calculus based integration problem, we are
able to retrieve standard representations of the forms of equations
and apply it to the problem to simplify it and solve it. Hence this
question is one of the questions that is central to understanding
cognition.
\item How do various processes act on this knowledge and how do they
achieve the effect they intend to achieve? This question is
significant because its answer explains the techniques of deduction
and inference we use to solve
problems on an everyday basis; this could be as simple and routine as diagnosing a
light bulb is not working and replace it, or perhaps the techniques we
use when solving a crossword puzzle.
% TODO Work in this, explain point below in further detail
\item How can these processes and structures be manifested in the real
world? This is summarized best by Bogdan\cite{Bogdan:1993aa}
\begin{quote}
It takes a real system, made of physical bits and pieces, to 
instantiate cognitive structures and processes and run the program of 
cognition.
\end{quote}

\end{itemize}

These questions reflect the the rationalist \cite{Bogdan:1993aa}
approach towards cognitive  
science as indicated by rationalist thinkers such as Descartes and
Leibniz  and is characterized ever since with a
deep analysis and explanation of every level of cognition.  


% I have explained the main questions that are required by any
% cognitive architecture,

% Add Newell's criteria and show they are essentially relate to one of
% the top three questions.

\subsubsection{The classical perspective}

\subsubsection{The connectionist perspective}

\subsection{Cognitive Architectures}

\subsubsection{Newell's criteria}

\subsubsection{ACT-R}

\subsubsection{EPIC}

\subsubsection{SOAR} 
\label{ClassicalPerspective}
   
\chapter{Cognitive Architectures}
\label{chap-two}
The quest to understand the working of the human mind has spanned
many centuries starting with Plato when he asked, as in the words of
Noam Chomsky citing Bertrand Russell, ``How comes is that human beings
whose contacts of the world are brief, personal and limited, are
nevertheless able to know as much as they do.'' \cite{Bogdan:1993aa}

Cognitive science brings together the varied disciplines of
psychology, neuroscience, computer science, linguistics and philosophy in an
attempt to answer the above question, using information processes as a
means to emulate the algorithms and processes of the human
mind. Psychology, especially cognitive psychology contributes theories
on cognitive capacities, information processing capabilities and
perhaps most importantly it tries to provide hypothesis that theorize
about the overall picture of the human mind. Neuroscience, the study
of the nervous system,  provides a frame of reference against which
theories developed in cognitive science can be validated since it
deals with the brain at the lowest level. Secondly, it provides  knowledge
for developing an alternative architecture of the mind. Computer
Science contributes to the enterprise knowledge representation which is
used to develop theories to represent the way knowledge is stored;
artificial intelligence which is used to analyse and create methods
for problem solving; the theory of computation which is used as a
means to develop representations for cognitive
architectures.

The objective of this chapter is three-fold; firstly it aims
to provide a very brief introduction to the human cognitive
architecture from both the classical and connectionist perspective. Secondly
it discusses cognitive architecture in general and finally goes on
to compare some of the currently widely used cognitive architectures
such as SOAR, EPIC and ACT-R, to name a few.

\subsection{The architecture of the mind}
\label{mindArch}

% What am I trying to explain?

%I could talk about information processes and how they relate to
%cognitive science?

% I am trying to give the perspective provided by the classical and
% connectionist perspective 

% SO what do I need to explain as the introduction to the section

% Perhaps I could explain

\subsubsection{The classical perspective}

\subsubsection{The connectionist perspective}

\subsection{Cognitive Architectures}

\subsubsection{Newell's criteria}

\subsubsection{ACT-R}

\subsubsection{EPIC}

\subsubsection{SOAR} 
\label{ClassicalPerspective}
   